\chapter{Specifični zahtjevi}
\label{poglavlje3}
\section{Funkcionalni zahtjevi}
\subsection{Administracija zaposlenih}
Administracija zaposlenih se odnosi na dvije različite aktivnosti, a to su ažuriranje podataka o 
zaposlenom i kreiranje naloga. Obje ove aktivnosti može da vrši samo administrator.

Da bi se ažurirali podaci o zaposlenom, potrebno je odabrati njegov nalog iz liste svih ponuđenih naloga nakon 
čega se prikazuju svi podaci o zaposlenom. Nakon izbora opcije za izmjenu, otvara se forma za izmjenu podataka. 
Poslije unosa novih podataka potrebno je utvrditi da su podaci validni i ako jesu vrši se čuvanje podataka. 
U suprotnom se nevalidni podaci moraju ispraviti. 

Da bi se napravio novi nalog moraju se unijeti podaci o novom zaposlenom nakon čega se vrši validacija. Ako su 
svi podaci validni, nalog je kreiran i vrši se njegova aktivacija.

\subsection{Administracija distributera}
Administracija distributera se odnosi na dvije različite aktivnosti, a to su ažuriranje podataka 
o distributeru i dodavanje distributera. Obje ove aktivnosti može da vrši samo administrator.
 
Da bi se izvršilo ažuriranje distibutera potrebno je odabrati željenog distributera iz liste nakon čega se prikazuju  svi
podaci o distributeru. Nakon izbora opcije za izmjenu, otvara se forma sa 
podacima i vrše željene promjene. Prilikom potvrde unosa, izvršiće se validacija podataka i u zavisnosti od rezultata 
čuvanje podataka ili vraćanje korisnika na prethodnu formu radi korekcije navalidnih podataka.

Da bi se dodao novi distibuter unose se njegovi podaci nakon čega se vrši validacija. Ako su svi podaci validni, 
distributer je dodan i svi potrebni podaci sačuvani. U suprotnom se nevalidni podaci moraju ispraviti. 

\subsection{Upravljanje filmovima}
Upravljanje filmovima obuhvata više različitih aktivnosti. To su izmjena podataka, naručivanje filmova.

Da bi se izmjenili podaci o filmu potrebno je isti selektovati u listi svih filmova. Nakon toga se prikazuju podaci o filmu i 
izborom opcije za izmjenu otvara se forma u kojoj je podatke moguće promijeniti. Poslije izmjena se vrši validacija i, u zavisnosti
 od ishoda validacije, čuvanje podataka ili ispravljanje grešaka načinjenih prilikom unosa. 
 
Da bi se naručio film potrebno je izabrati željeni film iz liste filmova dostupnih za naručivanje. Nakon toga se prikazuju podaci o 
izabranom filmu i moguće je naručiti film. Broj filmova u listi se može smanjiti unosom ključnih pojmova pri čemu se prikazuju samo 
filmovi koji te pojmove sadrže.

\subsection{Unos filmova u sistem}
Ukoliko se želi dodati novi film, o njemu se unesu podaci, izvrši se validacija i, u zavisnosti od ishoda validacije, čuvanje podataka
 ili ispravljanje grešaka načinjenih prilikom unosa.

\subsection{Generisanje izvještaja}

\subsection{Administracija sala}
U okviru administracije sale moguće je izabrati jednu od dvije aktivnosti, a to su dodavanje nove sale i izmjena podataka o postojećim salama.

Prilikom dodavanja nove sale moraju se unijeti svi podaci o sali (ime, broj redova, broj sjedišta u redu). Takođe je moguće izabrati da li je sala aktivna ili neaktivna, tj. da li se može koristiti za prikazivanje projekcija ili ne.

Da bi se izvršila izmjena podataka o salama, potrebno je odbrati za koju salu se mijenjaju podaci i izvršiti željene izmjene, a nakon toga potvrditi izmjene. Prilikom potvrde  vrši se validacija i u zavisnosti od ishoda čuvanje podataka ili vraćanje na formu radi korekcije unesenih vrijednosti.

\subsection{Dodavanje projekcije}
Dodavanje nove projekcije može da izvrši samo upravnik. Pri tome se otvara forma na kojoj je moguće unijeti podatke o novoj 
projekciji (film, vrijeme, sala i slično). Nakon toga se provjerava da li je odabrana 
sala zauzeta u odabranom terminu. Ako nije onda se ta projekcija može dodati i karte za nju postaju dostupne u prodaji. U suprotnom 
se moraju izabrati ili drugi termin u kojem je sala dostupna ili sala koja je dostupna u odabranom terminu.

\subsection{Upravljanje projekcijama}
Upravljanje projekcijama se odnosi na izmjenu podataka o postojećim projekcijama. Iz liste svih projekcija se bira projekcija 
za koju se žele promijeniti podaci. Nakon prikaza podataka bira se opcija za izmjenu podataka i otvara forma u kojoj se 
željeni podaci mogu promijeniti. Prije čuvanja podataka se vrši validacija unesenih podataka.

\subsection{Rezervacija karata}
Da bi se izvršila rezervacija karata potrebno je odabrati projekciju. Nakon prikaza svih podatka o odabranoj projekciji, 
vrši se provjera slobodnih mjesta i ukoliko postoje mjesta koja su zadovoljavajuća za gledaoca, karte se rezervišu 
i odabrana mjesta se smatraju zauzetim.

\subsection{Poništavanje rezervacija}
Ukoliko je isteklo vrijeme za preuzimanje rezervacija, potrebno je odabrati opciju za poništavanje svih rezervacija za 
koje karte nisu kupljene pri čemu ta mjesta na projekcijama postaju dostupna. 
Ako neki gledalac želi da poništi svoju rezervaciju onda se iz liste bira rezervacija koja se želi poništiti i nakon potvrde poništavanja
ta mjesta postaju dostupna.

\subsection{Prodaja karata}
Kod prodaje karata postoje dvije opcije:
\begin{itemize}
\item gledalac ima rezervaciju;
\item gledalac nema rezervaciju.
\end{itemize}

Ukoliko gledalac ima rezervaciju, otvara se prozor sa listom gdje se bira rezervacija za koju se kupuje karta. Nakon toga se potvrđuje kupovina i izdaje karta.

Ako rezervacija ne postoji, bira se projekcija za koju se kupuje karta nakon čega se prikazuje izgled sale. Označena su mjesta za koja su karte prodane i rezervisane. Ukoliko se nađu zadovoljavajuća mjesta za gledaoca, ona se selektuju, potvrđuje se prodaja karte, nakon čega se ista i izdaje.

\subsection{Poništavanje karata}
Nakon izbora opcije za poništavanje karata, potrebno je unijeti broj karte koja se poništava. Nakon toga se potvrđuje poništavanje. Ukoliko takva karta postoji, poništavanje je izvršeno, karta se vraća u prodaju i oslobađaju se mjesta. Ukoliko karta ne postoji, dobija se obavještenje o tome.

\section{Zahtjevi performansi}
Softver treba da izvršava operacije za što kraće vrijeme. Na sve akcije koje korisnik može poduzeti, sistem, u 95\% slučajeva, 
odgovara u roku 2-3 sekunde.

\section{Ograničenja dizajna}
Softver bi trebao da bude dizajniran tako da zadovoljava osnovne principe HCI,
tj. trebao bi da bude što jednostavnije dizajniran, bez pretjerivanja u bojama,
slikama, jer sve treba da bude umjereno, zatim, komande i tekstualna polja i
ostalo što će se nalaziti na prozorima, treba da bude grupisano i poravnato, kako
bi korisnik mogao lakše da se snađe.

Formati izvještaja bi trebali da budu onakvi na kakve su korisnici navikli, znači u
gornjem lijevom uglu bi trebao da se nalazi logo fakulteta, naziv, adresa, kontakt
telefon i e-mail adresa, ispod toga, najbolje na sredini, trebalo bi da se nalazi
naslov izvještaja, pa ispod njega sadržaj, i u donjem lijevom uglu naznačeno
mjesto za potpis službenika i mjesto za pečat.

\section{Atributi softverskog sistema}
\begin{itemize}
\item Softver bi trebao da bude proširiv, što znači da bi se u slučaju potrebe za novim funkcijama, dodavanje 
istih moglo da se izvrši u što kraćem roku.
\item Softver bi trebao da bude pouzdan, tj. da pruža korisniku tačne informacije.
\item Softver bi trebao da bude u potpunosti funkcionalan na Windows operativnom sistemu.
\item Softver bi trebao da sadrži i jedan ReadMe dokument koji će sadržati informacije o minimalnim konfiguracionim 
zahtjevima kao i druge informacije koje bi mogle biti od koristi.
\item Sam kod softvera bi trebao da bude dobro dokumentovan kako bi se omogućile lake izmjene.
\end{itemize}

\section{Ostali zahtjevi}
