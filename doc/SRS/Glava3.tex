\chapter{Specifični zahtjevi}
\label{poglavlje3}
\section{Eksterni interfejsi}
% This should be a detailed description of all inputs into and outputs from the software system. It should
% complement the interface descriptions in 5.2 and should not repeat information there.
% It should include both content and format as follows:
% Name of item;
% Description of purpose;
% Source of input or destination of output;
% Valid range, accuracy, and/or tolerance;
% Units of measure;
% Timing;
% Relationships to other inputs/outputs;
% Screen formats/organization;
% Window formats/organization;
% Data formats;
% Command formats;
% End messages.

\section{Funkcije softvera}
% Functional requirements should deÞne the fundamental actions that must take place in the software in
% accepting and processing the inputs and in processing and generating the outputs. These are generally listed
% as ÒshallÓ statements starting with ÒThe system shallÉÓ
% These include
% Validity checks on the inputs
% Exact sequence of operations
% Responses to abnormal situations, including
% 1) Overßow
% 2) Communication facilities
% 3) Error handling and recovery
% Effect of parameters
% Relationship of outputs to inputs, including
% 1) Input/output sequences
% 2) Formulas for input to output conversion
% It may be appropriate to partition the functional requirements into subfunctions or subprocesses. This does
% not imply that the software design will also be partitioned that way.
