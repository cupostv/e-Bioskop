\chapter{Uvod}
% The introduction of the SRS should provide an overview of the entire SRS.
\pagenumbering{arabic}
\section{Namjena}
% This subsection should
% Delineate the purpose of the SRS;
% Specify the intended audience for the SRS.

Ova specifikacija opisuje softverski proizvod - aplikaciju za rad bioskopa. Osnovna namjena ovog dokumenta je specifikacija zahtjeva informacionog sistema bioskopa.
Ova specifikacija je namjenjena naručiocu (kupcu) proizvoda, da bi se ustanovilo da li softver ispunjava njegove zahtjeve, kao i proizvođaču softvera za pregled svega što je potrebno implementirati u sistemu. Specifikacija opisuje ciljeve projekta, parametre, korisnički interfejs, zahtjeve i pitanje dizajna.

Ova specifikacija je napisana po preporuci \textit{IEEE Std 830-1998},
\section{Područje}
% This subsection should
% Identify the software product(s) to be produced by name (e.g., Host DBMS, Report Generator, etc.);
% Explain what the software product(s) will, and, if necessary, will not do;
% Describe the application of the software being speciÞed, including relevant beneÞts, objectives, and
% goals;
% Be consistent with similar statements in higher-level speciÞcations (e.g., the system requirements
% speciÞcation), if they exist.

Naziv softverskog proizvoda je "\naziv". Ovaj softver bi trebao da omogućava:
\begin{itemize}
  \item prodaju, rezervaciju karata;
  \item poništavanje rezervacije karata;
  \item unos filmova i termina projekcije;
  \item pregled stanja arhive filmova;
  \item generisanje izvještaja o poslovanju bioskopa;
  \item upravljanje projekcijama;
  \item administraciju korisnika.
\end{itemize}

Ovaj softver je namjenjen korištenju u bioskopima i trebao bi da olakša i ubrza rad zaposlenih u bioskopu.

Ne predstavlja proširenje postojećeg informacionog sistema, već predstavlja cjelinu za sebe.

\section{Definicije,akronimi i skraćenice}
%This subsection should provide the deÞnitions of all terms, acronyms, and abbreviations required to properly
% interpret the SRS. This information may be provided by reference to one or more appendixes in the SRS or
% by reference to other documents.
\subsection{Definicije}

\begin{itemize}
\item \textbf{kupac (naručilac) softvera} - osoba ili organizacija, koja naručuje (kupuje) gotov softverski proizvod od proizvođača, ili softver koji je izrađen po njegovom zahtjevu. 
\item \textbf{proizvođač softvera} - osoba ili organizacija koja izrađuje (programira) softver.
\item \textbf{radnik} - zaposleni u bioskopu koji obavlja prodaju karata, puštanje filmova itd.
\item \textbf{administrator} - zaposleni u bisoskopu koji je zadužen za održavanje sistema, kao i vođenje računa o ažurnosti podataka u sistemu;
\item \textbf{gledalac} - osoba koja dolazi u bioskop da gleda film.

\item \textbf{karta} - potvrda koja gledaocu omogućuje da uđe u projekcionu salu i gleda film;

\item \textbf{projekciona sala} - prostorija u kojoj se projektuju (puštaju) filmovi;

\item \textbf{projekcija filma} - vremenski termin u kojem se u jednoj sali za projekciju prikazuje film;

\item \textbf{film} - sekvenca pokretnih slika, koja je namijenjena za prikazivanje u javnosti;

\item \textbf{izvještaj} - dokument koji generiše sistem i sadrži različite podatke o radu bioskopa;

\item \textbf{arhiva} - skladište podataka u kome se čuvaju podaci o filmovima koji više nisu aktuelni u bioskopu;

\item \textbf{\naziv } - naziv softverkog proizvoda za koji je napisana ova specifikacija;

\item \textbf{SAP Crystal Reports} - sistem za generisanje izvještaja;
\end{itemize}

\subsection{Skraćenice}
\begin{itemize}
\item \textbf{ETFBL} - Univerzitet u Banjoj Luci, Elektrotehnički fakultet;
\item \textbf{IEEE} - The Institute of Electrical and Electronics Engineers;
\item \textbf{SRS} - Software Requirements Specification (specifikaicja korisničkih zahtjeva);
\item \textbf{HCI} – Human Computer Interaction (interakcija čovjek – računar);
\item \textbf{PDF} - \textit{Portable document format} - jedan od često korištenih formata dokumenata;
\end{itemize}

\section{Reference}
%This subsection should
% Provide a complete list of all documents referenced elsewhere in the SRS;
% Identify each document by title, report number (if applicable), date, and publishing organization;
% Specify the sources from which the references can be obtained.
% This information may be provided by reference to an appendix or to another document.

\begin{itemize}
\item IEEE Recommended Practice for Software Requirements Specifications, IEEE Std 830-1998
\end{itemize}
\section{Pregled}
%This subsection should
% Describe what the rest of the SRS contains;
% Explain how the SRS is organized.

Drugo poglavlje sadrži opis perspektive sistema, funkcija proizvoda, karakteristika korisnika, odnosno sadrži generalne informacije o funkcionalnim zahtjevima i zahtjevima podataka za softver "\naziv", sa detaljnim informacijama o korisnicima i perspektivama proizvoda. Takođe, sadrži i informacije o ograničenjima, pretpostavkama i zavisnostima softvera.

Treće poglavlje sadrži više tehničkih informacija, koje proizvođači softvera trebaju, a opisuje specifične zahtjeve informacionog sistema.
